\section{Conclusión}
En este trabajo pudimos ver, como a través del formalismo DEVS, pudimos ver el comportamiento de una
casa conectada a la red y sus generadores.

El uso de datos reales de consumo, así como tambien de datos climaticos, y las especificaciones de los
generadores nos permitió observar el comportamiento del consumo de energía en un escenario similar al
de la vida real. Esto nos permite ver, que si conseguimos los datos necesarios, podríamos usar los 
resultados de la simulación, por ejemplo, como un test de factibilidad a la hora de decidir si invertir
en este tipo de instalaciones, así como también que tipo de equipos comprar.

Como trabajo a futuro, se prodría analizar diferentes instalaciones y también como afectan a la red 
la conexión de diferentes tipos casas, vendiendo y solicitando energía, por ejemplo, si las demandas y 
ventas coinciden y como afecta esto a la red (por ejemplo picos o caidas de tensión).
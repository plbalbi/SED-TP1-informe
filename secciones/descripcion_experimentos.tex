\section{Experimentación:}

\subsection{Elección de generadores}
Para calcular la energía generada por la celda solar, usamos la siguiente formula:

$$producedEnergy = \frac{radiation \times PeakPower \times Area}{1000}$$

Donde $PeakPower$ es un factor que depende del generador.

Buscamos las especificaciones de un equipo\footnote{https://www.fiasa.com.ar/DoWNloADs/FIASA-FOLLETO-GENERADOR-SOLAR.pdf}  
y tomamos uno con $PeakPower=100$. Consideramos entonces un panel solar de 1m$^2$ y con estos valores calculamos la
energía producida por este medio.

Para la turbina eólica buscamos un modelo teórico que explique los valores generados,
pero no lo puedimos ajustar con las especificaciones de turbinas, así que elegimos un
modelomy utilizamos una interpolación de manera que se ajuste a los valores 
provistos por el vendedor. \footnote{https://articulo.mercadolibre.com.ar/MLA-616433420-turbina-generador-eolico-aerogenerador-1200w-48v-enertik-\_JM}


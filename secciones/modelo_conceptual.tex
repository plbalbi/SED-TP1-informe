\section{Modelo Conceptual}

Como se mencionó en la introducción, el sistema a modelar va a corresponder a
un nodo de la red que implementa el concepto de SmartGrid. Nótese que se hace
referencia a un nodo, sin especificar si el mismo corresponde a una vivienda,
un edificio de departamentos, o una fábrica. Esto se debe que el modelo es lo
suficientemente genérico, que permitiría representar a cualquiera de los
anteriores, solo variando ciertos \textit{parámetros} del mismo. Por otro
lado, se modelará un elemento que represente a la red eléctrica en si, con la
que interactúa el nodo.

Para que un nodo pueda participar dentro una SmartGrid, el mismo debería
tener los siguiente sus-sistemas:
\begin{itemize}
    \item Algún mecanismo de generación de energía eléctrica, preferentemente
    sustentable (no basado en combustibles fósiles, ya que en esto se basa el
    motivante del sistema).
    \item Una batería, en la cual almacenar la energía producida.
    \item Un controlador, el cual tendrá la responsabilidad de interactuar con la red.
\end{itemize}

Para los generadores, se modelará dos clases de ellos: generador eólico, y
paneles solares. Estos consistirán en dos modelos atómicos sin mucha
complejidad. Tomarán como entrada el factor climático que los afecte, y
tendrán como salida la cantidad de energía que están generando en ese
instante. Esta última solo variará en caso que se produzca un cambio en el
estado climático del sitio donde se encuentran, es decir, si reciben un
mensaje con dicha novedad.

La batería también será implementada en un modelos atómico. En este caso,
será más compleja que los anteriores, ya que tendrá una serie de estados
internos que cambian dependiendo si se encuentra descargada, cargando o
siendo consumida. tendrá dos puertos de entrada, en los cuales se le
notificará la energía que está siendo producida y la energía solicitada por
el controlador. Por otro lado, tendrá un puerto de salida por el cual
notificará al controlador cuando se produzca un cambio de estado en la misma.

% TODO: Agregar modelo conceptual controlador
% TODO: Agregar modelo conceptual red

Una vez modelado un nodo, nos interesa ver de qué forma el mismo interactúa
con la red, tomando como entrada datos climáticos reales, y un perfil de
carga acorde con el de un hogar promedio\footnote{Estos detalles serán
explicados en la sección de experimentación.}. Simulando el comportamiento
del mismo a lo largo de un mes, en diferentes estaciones del año, se podrá
observar patrones de generación y compra de energía a la red. 

También, si fuera así, podría llegar a detectarse posibles mejoras al
mecanismo mediante el cual el controlador opta por qué fuente de energía
utilizar, o detectar cuellos de botella en la capacidad tanto de los
generadores como de la batería.
\documentclass[a4paper]{article}
\usepackage[width=15.5cm, left=3cm, top=2.5cm, right=2cm, left=2cm, height= 24.5cm]{geometry}
\usepackage{comment}
\usepackage[utf8]{inputenc}
\usepackage[spanish]{babel}
\usepackage{amsmath}
\usepackage{amssymb}
\usepackage{enumitem}
\usepackage{hyperref}
\usepackage[table]{xcolor}
\usepackage{caratulaMetNum}
\usepackage{verbatim}
\usepackage[draft]{todonotes}
\usepackage{multicol}
% \usepackage[firstpage=true]{background}
\usepackage{xr}
\usepackage{csvsimple}
\usepackage{xpatch}
\usepackage{booktabs}
\usepackage{soul}
\usepackage{bytefield}
\usepackage{listings}
\usepackage{caption}
\usepackage{graphicx}
\usepackage{float}
\usepackage[draft]{todonotes}
\usepackage{subcaption}

\usepackage{algorithm}
\usepackage{algorithmic}

\hypersetup{
	colorlinks = true,
    urlcolor = blue,
    linkcolor = red,
    citecolor = red
}

\usepackage{array} % Nuevo tipo de columna al que le podés especificar ancho
\newcolumntype{C}[1]{>{\centering\let\newline\\\arraybackslash}m{#1}}

\newcommand{\standardroman}[1]{\romannumeral\value{#1}}
\makeatletter
\xpatchcmd{\csv@breakline@kernel}{\roman}{\standardroman}{}{}
\xpatchcmd{\csv@current@col}{\roman}{\standardroman}{}{}
\xpatchcmd{\set@csv@autohead}{\roman}{\standardroman}{}{}
\xpatchcmd{\set@csv@head}{\roman}{\standardroman}{}{}
\xpatchcmd{\set@csv@nohead}{\roman}{\standardroman}{}{}
\makeatother	

\setlength{\parskip}{3mm}

% Magia para excluir los brainstorming del table of contents
\usepackage[toc,page]{appendix}
\newcommand{\stoptocwriting}{%
  \addtocontents{toc}{\protect\setcounter{tocdepth}{-5}}}
\newcommand{\resumetocwriting}{%
  \addtocontents{toc}{\protect\setcounter{tocdepth}{\arabic{tocdepth}}}}

\begin{document}
	\titulo{Descripción del tp}
	\fecha{\today}
	\materia{Simulación de Eventos Discretos}
	\titulo{Trabajo Práctico Número 1}
	\subtitulo{SmartGrids}

	\integrante{Alfredo Umfurer}{}{}
	\integrante{Pablo Balbi}{707/15}{pablo.l.balbi@gmail.com}

	\maketitle

	\tableofcontents
	\pagebreak

\section{Introducción} Hoy en día la demanda por solucionar los problemas
energéticos crece más y más. Con problemas, se hace referencia a la busqueda
y explotación de fuentes de energía eléctrica que sea sustentables, no
requieran combustibles fósiles, y su impacto ambiental sea el mínimos
posible.

Actualmente existen centrales de generación eléctrica sustentables, que
explotan la energía eólica, la solar, entre otras. Pero estas funcionana a
gran escala, aportando directamente a la red eléctrica local. Otra
posibilidad, es que cada usuario de la red eléctrica tenga la capacidad de
generar energía, a una pequeña escala. Esto da a la posibilidad que la
producción de energía eléctrica sea distribuida entre los usuarios.

En el segundo caso, en que cada usuario de la red eléctrica tiene la
capaicdad de producir, existe el caso en que el mismo generé más o menos de
lo que consume. En el primer caso, el mismo puede recurrir al proveedor de
energía local, para comprar el faltante, cosa que se realiza en la totalidad
del consumo hoy en día. El segundo punto da lugar a una posiblidad
interesante, que consiste en que el usuario puede \textbf{vender} el
excedente producido a la red eléctrica, quedando con cierta gananacia o saldo
a favor. Esto da lugar a una red de generación eléctrica distribuida e
inteligente, o
\textbf{SmartGrid}\footnote{\url{https://www.smartgrid.gov/the_smart_grid/smart_grid.html}}.

En este trabajo se va modelar y el funcionamiento de un nodo perteneciente a
un smartgrid. El mismo tendrá un generador eólica, un panel fotoltaíco, una
batería (en la cual se almacena la energía producida por ambos generadores),
y un controlador, que decidirá en qué momento es conveniente consumir la
energía producida, o comprar de la red.

\todo{Poner objetivos aca!} \pagebreak
\section{Modelo Conceptual}

Como se mencionó en la introducción, el sistema a modelar va a corresponder a
un nodo de la red que implementa el concepto de SmartGrid. Nótese que se hace
referencia a un nodo, sin especificar si el mismo corresponde a una vivienda,
un edificio de departamentos, o una fábrica. Esto se debe que el modelo es lo
suficientemente genérico, que permitiría representar a cualquiera de los
anteriores, solo variando ciertos \textit{parámetros} del mismo. Por otro
lado, se modelará un elemento que represente a la red eléctrica en si, con la
que interactúa el nodo.

Para que un nodo pueda participar dentro una SmartGrid, el mismo debería
tener los siguiente sus-sistemas:
\begin{itemize}
    \item Algún mecanismo de generación de energía eléctrica, preferentemente
    sustentable (no basado en combustibles fósiles, ya que en esto se basa el
    motivante del sistema).
    \item Una batería, en la cual almacenar la energía producida.
    \item Un controlador, el cual tendrá la responsabilidad de interactuar con la red.
\end{itemize}

Para los generadores, se modelará dos clases de ellos: generador eólico, y
paneles solares. Estos consistirán en dos modelos atómicos sin mucha
complejidad. Tomarán como entrada el factor climático que los afecte, y
tendrán como salida la cantidad de energía que están generando en ese
instante. Esta última solo variará en caso que se produzca un cambio en el
estado climático del sitio donde se encuentran, es decir, si reciben un
mensaje con dicha novedad.

La batería también será implementada en un modelos atómico. En este caso,
será más compleja que los anteriores, ya que tendrá una serie de estados
internos que cambian dependiendo si se encuentra descargada, cargando o
siendo consumida. tendrá dos puertos de entrada, en los cuales se le
notificará la energía que está siendo producida y la energía solicitada por
el controlador. Por otro lado, tendrá un puerto de salida por el cual
notificará al controlador cuando se produzca un cambio de estado en la misma.

% TODO: Agregar modelo conceptual controlador
% TODO: Agregar modelo conceptual red

Una vez modelado un nodo, nos interesa ver de qué forma el mismo interactúa
con la red, tomando como entrada datos climáticos reales, y un perfil de
carga acorde con el de un hogar promedio\footnote{Estos detalles serán
explicados en la sección de experimentación.}. Simulando el comportamiento
del mismo a lo largo de un mes, en diferentes estaciones del año, se podrá
observar patrones de generación y compra de energía a la red. 

También, si fuera así, podría llegar a detectarse posibles mejoras al
mecanismo mediante el cual el controlador opta por qué fuente de energía
utilizar, o detectar cuellos de botella en la capacidad tanto de los
generadores como de la batería.
\section{Especificación Formal}

\subsection{Modelos Atómicos}

\subsubsection{Panel solar y generador eólico}
Como fue descripto en el modelo conceptual (\cite{section:modeloConceptual}),
el modelo que representa al panel solar, y también al generador eólico son
simples. Ambos tendrán un puerto de entrada, el cual recibirá el valor de la
magnitud de clima que los afecta, siendo radiación en el caso del primero, y
la velocidad del viento para el segundo. Por otro lado, tendrán un puerto de
salida, por el cual notificarán a los modelos que tengan una conexión al
mismo, cual es la nueva cantidad de energía que están produciendo.

% TODO: Poner algo haciendo referencia a que DEVS puede manejar eventos (cambios en dichas magnitudes climáticas) a distintas frecuencias de ocurrencia. Leer en el Wainer como se explicaba esto.
% TODO: Poner las funciones de transición, y las funciones matemáticas con las cuales se calculará la potencia producida.

\subsubsection{Batería}
\subsubsection{Controlador}
\subsubsection{Carga}
\subsubsection{Red Eléctrica}

\subsection{Modelos Acoplados}

\subsubsection{Casa}
\subsubsection{Top Model}

\end{document}

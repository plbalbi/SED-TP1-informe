\documentclass[a4paper]{article}
\usepackage[width=15.5cm, left=3cm, top=2.5cm, right=2cm, left=2cm, height= 24.5cm]{geometry}
\usepackage{comment}
\usepackage[utf8]{inputenc}
\usepackage[spanish]{babel}
\usepackage{amsmath}
\usepackage{amssymb}
\usepackage{enumitem}
\usepackage{hyperref}
\usepackage[table]{xcolor}
\usepackage{caratulaMetNum}
\usepackage{verbatim}
\usepackage[draft]{todonotes}
\usepackage{multicol}
% \usepackage[firstpage=true]{background}
\usepackage{xr}
\usepackage{csvsimple}
\usepackage{xpatch}
\usepackage{booktabs}
\usepackage{soul}
\usepackage{bytefield}
\usepackage{listings}
\usepackage{caption}
\usepackage{graphicx}
\usepackage{float}
\usepackage[draft]{todonotes}
\usepackage{subcaption}

\usepackage{algorithm}
\usepackage{algorithmic}

\hypersetup{
	colorlinks = true,
    urlcolor = blue,
    linkcolor = red,
    citecolor = red
}

\usepackage{array} % Nuevo tipo de columna al que le podés especificar ancho
\newcolumntype{C}[1]{>{\centering\let\newline\\\arraybackslash}m{#1}}

\newcommand{\standardroman}[1]{\romannumeral\value{#1}}
\makeatletter
\xpatchcmd{\csv@breakline@kernel}{\roman}{\standardroman}{}{}
\xpatchcmd{\csv@current@col}{\roman}{\standardroman}{}{}
\xpatchcmd{\set@csv@autohead}{\roman}{\standardroman}{}{}
\xpatchcmd{\set@csv@head}{\roman}{\standardroman}{}{}
\xpatchcmd{\set@csv@nohead}{\roman}{\standardroman}{}{}
\makeatother	

\setlength{\parskip}{3mm}

% Magia para excluir los brainstorming del table of contents
\usepackage[toc,page]{appendix}
\newcommand{\stoptocwriting}{%
  \addtocontents{toc}{\protect\setcounter{tocdepth}{-5}}}
\newcommand{\resumetocwriting}{%
  \addtocontents{toc}{\protect\setcounter{tocdepth}{\arabic{tocdepth}}}}

\DeclareCaptionFormat{myformat}{#3}
\captionsetup[algorithm]{format=myformat}

\begin{document}
	\titulo{Descripción del tp}
	\fecha{\today}
	\materia{Simulación de Eventos Discretos}
	\titulo{Trabajo Práctico Número 1}
	\subtitulo{SmartGrids}

	\integrante{Alfredo Umfurer}{528/12}{alfredoumfurer@gmail.com}
	\integrante{Pablo Balbi}{707/15}{pablo.l.balbi@gmail.com}

	\maketitle

	\tableofcontents
	\pagebreak

\section{Introducción} Hoy en día la demanda por solucionar los problemas
energéticos crece más y más. Con problemas, se hace referencia a la busqueda
y explotación de fuentes de energía eléctrica que sea sustentables, no
requieran combustibles fósiles, y su impacto ambiental sea el mínimos
posible.

Actualmente existen centrales de generación eléctrica sustentables, que
explotan la energía eólica, la solar, entre otras. Pero estas funcionana a
gran escala, aportando directamente a la red eléctrica local. Otra
posibilidad, es que cada usuario de la red eléctrica tenga la capacidad de
generar energía, a una pequeña escala. Esto da a la posibilidad que la
producción de energía eléctrica sea distribuida entre los usuarios.

En el segundo caso, en que cada usuario de la red eléctrica tiene la
capaicdad de producir, existe el caso en que el mismo generé más o menos de
lo que consume. En el primer caso, el mismo puede recurrir al proveedor de
energía local, para comprar el faltante, cosa que se realiza en la totalidad
del consumo hoy en día. El segundo punto da lugar a una posiblidad
interesante, que consiste en que el usuario puede \textbf{vender} el
excedente producido a la red eléctrica, quedando con cierta gananacia o saldo
a favor. Esto da lugar a una red de generación eléctrica distribuida e
inteligente, o
\textbf{SmartGrid}\footnote{\url{https://www.smartgrid.gov/the_smart_grid/smart_grid.html}}.

En este trabajo se va modelar y el funcionamiento de un nodo perteneciente a
un smartgrid. El mismo tendrá un generador eólica, un panel fotoltaíco, una
batería (en la cual se almacena la energía producida por ambos generadores),
y un controlador, que decidirá en qué momento es conveniente consumir la
energía producida, o comprar de la red.

\todo{Poner objetivos aca!} \pagebreak
\section{Modelo Conceptual}

Como se mencionó en la introducción, el sistema a modelar va a corresponder a
un nodo de la red que implementa el concepto de SmartGrid. Nótese que se hace
referencia a un nodo, sin especificar si el mismo corresponde a una vivienda,
un edificio de departamentos, o una fábrica. Esto se debe que el modelo es lo
suficientemente genérico, que permitiría representar a cualquiera de los
anteriores, solo variando ciertos \textit{parámetros} del mismo. Por otro
lado, se modelará un elemento que represente a la red eléctrica en si, con la
que interactúa el nodo.

Para que un nodo pueda participar dentro una SmartGrid, el mismo debería
tener los siguiente sus-sistemas:
\begin{itemize}
    \item Algún mecanismo de generación de energía eléctrica, preferentemente
    sustentable (no basado en combustibles fósiles, ya que en esto se basa el
    motivante del sistema).
    \item Una batería, en la cual almacenar la energía producida.
    \item Un controlador, el cual tendrá la responsabilidad de interactuar con la red.
\end{itemize}

Para los generadores, se modelará dos clases de ellos: generador eólico, y
paneles solares. Estos consistirán en dos modelos atómicos sin mucha
complejidad. Tomarán como entrada el factor climático que los afecte, y
tendrán como salida la cantidad de energía que están generando en ese
instante. Esta última solo variará en caso que se produzca un cambio en el
estado climático del sitio donde se encuentran, es decir, si reciben un
mensaje con dicha novedad.

La batería también será implementada en un modelos atómico. En este caso,
será más compleja que los anteriores, ya que tendrá una serie de estados
internos que cambian dependiendo si se encuentra descargada, cargando o
siendo consumida. tendrá dos puertos de entrada, en los cuales se le
notificará la energía que está siendo producida y la energía solicitada por
el controlador. Por otro lado, tendrá un puerto de salida por el cual
notificará al controlador cuando se produzca un cambio de estado en la misma.

% TODO: Agregar modelo conceptual controlador
% TODO: Agregar modelo conceptual red

Una vez modelado un nodo, nos interesa ver de qué forma el mismo interactúa
con la red, tomando como entrada datos climáticos reales, y un perfil de
carga acorde con el de un hogar promedio\footnote{Estos detalles serán
explicados en la sección de experimentación.}. Simulando el comportamiento
del mismo a lo largo de un mes, en diferentes estaciones del año, se podrá
observar patrones de generación y compra de energía a la red. 

También, si fuera así, podría llegar a detectarse posibles mejoras al
mecanismo mediante el cual el controlador opta por qué fuente de energía
utilizar, o detectar cuellos de botella en la capacidad tanto de los
generadores como de la batería.
\section{Especificación Formal}

\subsection{Modelos Atómicos}

\subsubsection{Panel solar y generador eólico}
Como fue descripto en el modelo conceptual (\cite{section:modeloConceptual}),
el modelo que representa al panel solar, y también al generador eólico son
simples. Ambos tendrán un puerto de entrada, el cual recibirá el valor de la
magnitud de clima que los afecta, siendo radiación en el caso del primero, y
la velocidad del viento para el segundo. Por otro lado, tendrán un puerto de
salida, por el cual notificarán a los modelos que tengan una conexión al
mismo, cual es la nueva cantidad de energía que están produciendo.

% TODO: Poner algo haciendo referencia a que DEVS puede manejar eventos (cambios en dichas magnitudes climáticas) a distintas frecuencias de ocurrencia. Leer en el Wainer como se explicaba esto.
% TODO: Poner las funciones de transición, y las funciones matemáticas con las cuales se calculará la potencia producida.

\subsubsection{Batería}
\subsubsection{Controlador}
\subsubsection{Carga}
\subsubsection{Red Eléctrica}

\subsection{Modelos Acoplados}

\subsubsection{Casa}
\subsubsection{Top Model}
% !TeX root = ../main.tex

\section{Simulación y análisis}

\subsection{Eventos} Nuestro modelo tiene la característica de alimentarse de
datos externos para simular los efectos climáticos en la generación de
energía, y las variaciones en el consumo de un hogar. Es por eso, que en vez
de utilizar datos producidos aleatoriamente exclusivamente para llevar a cabo
este trabajo, optamos por utilizar fuentes de datos disponibles en internet,
pero que corresponden a casos reales.

En primer, como fuente de datos de clima, utilizamos un dataset de
Kaggle\footnote{\url{https://www.kaggle.com/runphilrun/hi-seas-solar-radiation-prediction}},
una plataforma para abierta para compartir fuentes de datos, y llevar a cabo
competencias en ciencias de datos. Los mismos fueron recolectados por
NASA\footnote{\url{https://hi-seas.org/}}, en dos misiones llevadas a cabo en
una instalación ubicada en la base de un volcán en Hawaii, Estados Unidos. La
expedición es llamada HI-SEAS (\textit{Hawai’i Space Exploration Analog and
Simulation}), y los datos corresponden a las misiones IV y V (septiembre a
diciembre de 2016).

La fuente de datos posee diversas mediciones meteorológicas, pero de estas
solo nos interesa la radiación solar, medida en $\frac{Watts}{m^2}$; y la
velocidad del viento, medida en $\frac{km}{hora}$. Todas estas mediciones
fueron realizadas con una frecuencia de cinco minutos.

Por otro lado, la fuente de datos utilizada para el consumo eléctrico
corresponde a \textit{University of California,
Irvine}\footnote{\url{https://archive.ics.uci.edu/ml/datasets/Individual+household+electric+power+consumption}},
y fueron tomadas en una casa en Francia, entre diciembre de 2006 y 2010. El
dato que extrajimos de esta fuente es la potencia activa consumida, medida
cada un minuto, y cuya unidad es $Watts$. Algo importante a notar, que puede
ser posible gracias a la asincronía que soporta DEVS en los eventos que
suceden, es que $1,25\%$ de las mediciones son nulas (debido a algún
inconveniente en la lectura de las mismas), las cuales fueron omitidas, y por
consiguiente, no incluidas entre los eventos disparados en la simulación.

Los eventos fueron procesados y combinados en un solo archivo de eventos. El
procesamiento consistió en: \begin{itemize} \item Conversión de las unidades
originales de las fuentes de datos a las utilizadas en el modelo. \item
Normalización de las fechas asignadas a cada medición, para que las mismas
arranquen en 00:00:00:00 (en tiempo DEVS, el inicio de la simulación), y
conversión de las mismas al formato de tiempo utilizado por el simulador,
respetando la diferencia relativa entre los eventos. Para esto se tomo la
fecha 00:00:00:00 como la hora 00:00 AM, del primer día simulado. \item
Inclusión de ambas fuentes de datos en un mismo archivo de eventos. Resultó
interesante descubrir que el simulador ignora el orden en el que lee los
eventos del archivo, ya que después al programarlos para ser disparados los
reordena, lo cual permite armar el archivo no teniendo que ordenar los
eventos manualmente. \end{itemize} % TODO: Descripción cuantitativa de los
datos?

\subsection{Ambiente de Simulación y experimentación} El procedimiento que
utilizamos para llevar a cabo la simulación y analizar los datos, consiste
en, teniendo el archivo con los eventos procesados, simular la cantidad de
tiempo deseada (una semana en la mayor parte de los análisis). Luego,
utilizando los resultados obtenidos de la simulación (puertos de salida del
\textit{top model}, y logs (no propios de DEVS, sino instrumentados en el
modelos en si) obtenidos de alguna de los modelos), se realizó un análisis
cuantitativo y cualitativo.

Para el análisis de los datos, se utilizo \textit{Jupyter Notebooks}, junto
con diversas bibliotecas de \textit{Python}, tanto para lectura, filtrado y
transformación de los datos; como para graficar y obtener medidas
estadisticas.

\subsection{Desarrollo} El desarrollo esta organizado de la manera en la que
fueron sucediendo las simulaciones realizadas (se incluye en cada una las
configuraciones pasadas al simulador, de forma de poder replicarlas
fácilmente\footnote{Todos los comandos son tomando como directorio base: \textit{TP\_ROOT/simulation/src}.}; y el correspondiente \textit{notebook} con el análisis). Cada
sección tiene algunas conclusiones parciales, y construye sobre la anterior.

\subsubsection{¿Cuál medio de generación aporta más?}
% ./bin/cd++ -e../eventGeneration/mergedData.ev -mmodels/solarAndWindControllerModel.ma -oout/solarAndWind1Week -lout/solarAndWind1Week.log -t168:00:00:00
\section{Conclusión}
En este trabajo pudimos ver, como a través del formalismo DEVS, pudimos ver el comportamiento de una
casa conectada a la red y sus generadores.

El uso de datos reales de consumo, así como también de datos climáticos, y las especificaciones de los
generadores nos permitió observar el comportamiento del consumo de energía en un escenario similar al
de la vida real. Esto nos permite ver, que si conseguimos los datos necesarios, podríamos usar los 
resultados de la simulación, por ejemplo, como un test de factibilidad a la hora de decidir si invertir
en este tipo de instalaciones, así como también que tipo de equipos comprar.

Como trabajo a futuro, se podría analizar diferentes instalaciones y también como afectan a la red 
la conexión de diferentes tipos casas, vendiendo y solicitando energía, por ejemplo, si las demandas y 
ventas coinciden y como afecta esto a la red (por ejemplo picos o caídas de tensión).

\end{document}

\section{Introducción} Hoy en día la demanda por solucionar los problemas
energéticos crece más y más. Con problemas, se hace referencia a la busqueda
y explotación de fuentes de energía eléctrica que sea sustentables, no
requieran combustibles fósiles, y su impacto ambiental sea el mínimos
posible.

Actualmente existen centrales de generación eléctrica sustentables, que
explotan la energía eólica, la solar, entre otras. Pero estas funcionana a
gran escala, aportando directamente a la red eléctrica local. Otra
posibilidad, es que cada usuario de la red eléctrica tenga la capacidad de
generar energía, a una pequeña escala. Esto da a la posibilidad que la
producción de energía eléctrica sea distribuida entre los usuarios.

En el segundo caso, en que cada usuario de la red eléctrica tiene la
capaicdad de producir, existe el caso en que el mismo generé más o menos de
lo que consume. En el primer caso, el mismo puede recurrir al proveedor de
energía local, para comprar el faltante, cosa que se realiza en la totalidad
del consumo hoy en día. El segundo punto da lugar a una posiblidad
interesante, que consiste en que el usuario puede \textbf{vender} el
excedente producido a la red eléctrica, quedando con cierta gananacia o saldo
a favor. Esto da lugar a una red de generación eléctrica distribuida e
inteligente, o
\textbf{SmartGrid}\footnote{\url{https://www.smartgrid.gov/the_smart_grid/smart_grid.html}}.

En este trabajo se va modelar y el funcionamiento de un nodo perteneciente a
un smartgrid. El mismo tendrá un generador eólica, un panel fotoltaíco, una
batería (en la cual se almacena la energía producida por ambos generadores),
y un controlador, que decidirá en qué momento es conveniente consumir la
energía producida, o comprar de la red.

\todo{Poner objetivos aca!}